\documentclass[10pt]{article}
\usepackage{color}
\usepackage{palatino}

\topmargin=-0.5in
\oddsidemargin=0in
\evensidemargin=0in
\textwidth=6.5in
\textheight=9.0in

\pagestyle{empty}

\newcommand{\todo}[1]{\textcolor{red}{[#1]}}

\begin{document}
 \begin{Large}
\begin{center}
Broadening Participation in Computing
\end{center}
\end{Large}

Research on network design automation is poised to broaden participation in computing at three distinct, complementary levels:
\begin{enumerate}
\item \textbf{Direct Impact:} the participation impact of the research outputs, expanding network access which is a prerequisite for virtually all computing today.
\item \textbf{Community Education:} the potential of the research to involve and educate a broader community.
\item \textbf{Active Participation:} the active participation of a broader population in the research itself.
\end{enumerate}

\begin{table*}
\centering
\begin{tabular}{l|p{1.6in}|p{1.6in}|p{1.6in}|}
\cline{2-4}
 & \textbf{Direct Impact} & \textbf{Community Education} & \textbf{Active Participation} \\ \hline
\multicolumn{1}{|l|}{\textbf{Attitudes}} & Valuing and connecting neglected communities (A) & Participatory design (B) & Local research recruiting (C) \\ \hline
\multicolumn{1}{|l|}{\textbf{Infrastructure}} & Broadened network access (D) & Open source tools (E) & Language refinement (F) \\ \hline
\multicolumn{1}{|l|}{\textbf{Curriculum}} & NDA coursework (G) & Community workshops on NDA (H) & Curriculum design (I) \\ \hline
\multicolumn{1}{|l|}{\textbf{People}} & Mentorship (J) & Incubators (K) & Broader hiring and recruiting (L) \\ \hline
\end{tabular}
\caption{NDA Broadening Participation Activities.}
\label{t:bpc-impacts}
\end{table*}

We take inspiration from past fora on BPC (e.g., CRA Snowbird) and seek to broaden participation along four axes, which intersect the three levels we describe above: attitudes, infrastructure, curriculum, and personnel.  As a result, we aim for 12 BPC impacts, which we summarize in Table~\ref{t:bpc-impacts}.  Some of these impacts are natural consequences of our research agenda, as we will descibe below, and others are explicit efforts we will undertake to broaden participation in ways that are aligned with the proposed research agenda.

Changing attitudes is always a subtle but necessary task in broadening participation.  We plan to achieve this by, first and foremost, valuing and connecting neglected communities (A) by having them as a focus in our work.  As we have indicated throughout our proposal, we believe previously-neglected communities and the engineering and research challenges that arise in their context are worthy of study, and this shift in attitude is an important first step in broadening participation.  We plan to take this one step further by engaging in participatory design (B) in such communities (as Raghavan has done in the past in the context of rural WISP networks), engaging both technical and non-technical stakeholders in the process of designing a purpose-built network that meets the community's needs and leverages NDA.  Finally, we as a research community often neglect the recruiting of new researchers (students, staff, etc.) from marginalized communities such as communities of color and poor rural areas, despite the burgeoning talent they have; to begin to address this issue and to express our attitude that such communities have much to contribute, we plan to travel to give talks (e.g., at local community colleges) to recruit new students and staff (C) to work with us on NDA.

Today, the first and most important prerequisite for broadening participation in computing is access to the Internet.  There are, unfortunately, too many places in the United States that today still have little to no broadband Internet access and as a result are completely shut out from computer science (both research and engineering).  Our work on NDA will have a direct impact on expanding Internet access (D) in such communities, as we have outlined previously.
Beyond this, recall that we aim for the same with our \emph{network scripting} libraries: open-source software (E) that aspiring operators in rural areas and developing countries can use both to learn NDA techniques and to use in the field.  Operators of rural networks, because they are called upon to do many tasks, are likely to find useful more familiar programming paradigms, rather than sophisticated specification languages.  We aim to provide these operators with the ability to easily produce \emph{network scripts} that automate common tasks.  Recall that we will couple network scripting with a \emph{simulator} that can be used by rural operators to test the impact of scripting commands on models of real networks, including their own. This can serve both as an educational tool and to help assess impact of configuration changes pre-deployment.  Further, we plan to engage such operators in improving the infrastructure (tools, languages, etc.), directly engaging them in language design and refinement (F).

Once we have this in place, we plan to develop curriculum around NDA (G) at the undergraduate and graduate levels, incorporating the lessons of operators in the field.  As part of this coursework, we will aim to bring in (virtually or in person) operators and community members from marginalized regions to involve them in the education of students, to ensure that students are keenly aware of the challenges faced in non-traditional networks.   We will also specifically engage students from underrepresented groups in these courses.  At UCLA this will be facilitated by the Computer Science Department's affiliation with the Building, Recruiting, And Inclusion for Diversity (BRAID) initiative, led by the Anita Borg Institute and Harvey Mudd College. As part of that initiative, Millstein 
created a new introductory course in computing targeted specifically to students with no prior programming experience, which he taught for the first time in Fall 2017.  More than half of the enrolled students in the first offering were female, and feedback through anonymous surveys and course evaluations has been strongly positive.  We will leverage these existing connections and the broader BRAID initiative to engage underrepresented groups with our curriculum.

One way we plan to engage the community is to hold community workshops in Los Angeles on new ways to design and build networks (H), building upon our engagement with the community from activities (C) and (K).  We see these workshops as engaging a new generation of underrepresented young people in the promise of computer networking, and introducing them to modern techniques.  For too long the avenues available to such students have been now-archaic vocational training such as Cisco certification; our aim is to spark an interest in modern approaches to networking, setting up new pathways and possibilities for young people.

Finally, what matters most in broadening participation is a human focus: a focus on the people who we hope to engage in our research community.  To help reach underprepresented minorities and women to be network operators and networking researchers, we plan to undertake several efforts.  First, as a direct impact of our work, we must engage with and mentor existing network operators from disadvantaged communities (J).  This is something we have already undertaken in prior work, and have active ongoing efforts; Raghavan is advising a new urban WISP that aims to connect poor communities in Los Angeles County and another rural farming community in Northern California that has long been disconnected.  Further, we have developed a program through which we can create ``community-network incubators'' (K).  Such an incubator, like a startup incubator, would aim to identify new network operators or others interested in networking but who have few connections or resources, and provide them access to the PIs and the broader networking community with the aim of helping them materially and intellectually achieve their goals (e.g., of starting a new network, of joining the networking research community, etc.).  We have begun to develop a plan for such an incubator and plan to put it into action in the context of NDA.
Finally, broader hiring and student recruiting is a key to broadening participation (L).  The PIs have recruited broadly, with quite some success, and plan to continue these efforts. For example, Govindan and Raghavan currently co-advise one female PhD student; Govindan and Millstein previously co-advised one female PhD student, now an Engineering Manager at Microsoft Azure; Millstein currently advises one female PhD student; two of three of Raghavan's PhD students are women along with one undergraduate student; and at USC we are involved in mentoring in the USC NAI-STEM program for first-generation college students of underrepresented backgrounds from South and East LA.

Each of the above styles of engagement require different metrics for impact evaluation.  To evaluate Direct Impact, we can examine the number of people whose access to computing, networking, or NDA in particular was enhanced by our efforts (either by us or by others who we enabled).  To evaluate Community Education, we can examine the level of participation in community events that we hold and the online engagement in our efforts (e.g., in open-source codebases we develop).  To evaluate Active Participation, we can examine both quantitatively and qualitatively the experiences of those who come from underrepresented backgrounds and their engagement in networking and computing as a result of our research on NDA and our BPC efforts.

\end{document}
