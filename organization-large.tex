\section{Leadership, Collaboration and Outreach}
\label{sec:organization}

\paragraph*{Center for Network Design Automation.}
%
To facilitate broader collaboration with industry and other faculty members, we will establish the {\em Center for Network Design Automation} (CNDA) at UCLA jointly with 
USC with George Varghese as Director at UCLA and Ramesh Govindan as 
co-Director at USC. The UCLA center will be modeled after earlier successful centers at UCLA such as CENS (which established sensor networks) which was also 
joint with USC. 

\paragraph*{Academic and Industry Engagement.}
%
This project has the potential to transform academia, by helping to establish a new field, and industry, by revolutionizing the way that networks are built and operated. To achieve this impact, however, requires engaging with stakeholders in each of these communities beyond the PIs.  We have access to stakeholders at Microsoft, Google, Facebook, and Cisco to allow experimental validation at scale via interns. We will leverage our long-standing collaboration with MSR researchers in verification (Bjorner), software engineering (Godefroid), and networking (Mahajan).  We will engage with other researchers such as Foster, Rexford, and Walker via the Cornell-Princeton Center.

\paragraph*{Open-Source Software.}
%
To create, nurture, and grow a community of researchers working in Network Design Automation, we will encourage personnel to make software prototypes, mathematical models, topologies, traffic data, and other data sets associated with this project available on the project website under an open-source license. UCLA, where 4 of the PIs are based already has policies in place that permit release of software developed under sponsored research agreements. The PIs have a strong track record of releasing software prototypes and open data sets. 

\paragraph*{Collaborative Student Advising.}
%
Since we are in Los Angeles, we will strive to have at least half the students
supported be jointly advised by a PI in UCLA and USC.  It is students who 
are ultimately the bees that cross-pollinate researchers.  We note that Govindan
(USC) and Millstein (UCLA) have already demonstrated the effectiveness of this
strategy in their initial work on network verification that produced tools like
Batfish  (Ari Fogel) and the PIC Interoperability Tool (Luis Pedrosa).  Both Ari
and Luis were jointly advised and the student meetings were both in USC
and UCLA.



\paragraph*{Tutorials and Mentoring Under-Represented Minorities.}
%
We will develop tutorials on NDA and present them at professional meetings such as POPL and SIGCOMM. PI Varghese has a strong track record of organizing such meetings including tutorials at FMCAD '13, SIGCOMM '15, and a session at the 2015 Microsoft Faculty Summit. 

\paragraph*{Annual Conference.}
%
To catalyze broader interest in NDA, we plan to organize an annual week-long conference, which will rotate between sites.  We plan to invite leading network architects and operators to learn about the problems they are facing and explore how NDA can help. We will attempt to co-locate these workshops with existing events such as NANOG, ONS, or the P4 Workshop. The conference will provide an opportunity to give a status update on the project itself, to hear from outsiders working in related areas, and to allow an advisory  committee of experts to meet and review our progress each year.  We will invite the networking leads at Google, Microsoft, Forward Networks, and  Cisco/Candid or their representatives (see letters) to serve on this committee to ensure we focus on problems that matter.

\paragraph*{Outreach to the Developing World.}
%
The developing world is on the verge of being transformed by networked services.  We believe the canvas of developing countries presents a unique opportunity for achieving broad adoption of our ideas and having positive societal impact.  PI Barath Raghavan at USC will lead this effort.  In addition, Raghavan will look to Jay Chen (NYU) and David Hutchful (Grameen) who have deep experience in networking in African contexts and who have offered to provide us guidance on aligning our research with impact on the ground. Beyond network scripting, we envision a User Interface (UI), one usable from a mobile device in the field, that, for example, makes it easy to add a new upstream provider or delete a link, obviating the need for more abstract specifications of regular expressions and route preferences.  The Celerate system developed by Raghavan already provides a UI (that is used only for extracting, recording, and displaying the current topology and other network parameters); we envision extending it to synthesize configuration commands for BGP or low-level Linux route commands, using the idea of {\em network scripting} described in the educational section.

\paragraph*{Research team.}  We have assembled a team from across $2$ institutions on both coasts who wish to work together to develop the NDA vision {\em and} innovate in their own fields---{\em the special structure of the networking domain often reveals new approaches}. {\bf Millstein} is an expert in programming language design~\cite{expanders,jpred,planb}, static program analysis, and verification~\cite{Ball:2001,rhodium,javacop,DBLP:conf/cav/LesaniMP14}. For the last 10 years, he has worked with {\bf Govindan} and Ratul Mahajan at MSR to automatically identify security and interoperability errors~\cite{DBLP:conf/sigcomm/KothariMMGM11, DBLP:conf/nsdi/PedrosaFKGMM15}, and verify network configurations~\cite{batfish,propane}.
{\bf Govindan} managed the early development of the  Routing Policy Specification Language (RPSL) which is used to coordinate routing policy across small providers.    {\bf Varghese} won the SIGCOMM Award for his work in making the Internet faster~\cite{confluence,varghesebook} but has worked on bottom-up network verification for the last 6 years~\cite{hsa,nod,surgeries}.  Further, his approach  to interdisciplinary thinking via confluences~\cite{confluence} and his experience collaborating with theoreticians, architects, and database researchers will help meld this team into a synergistic unit. Varghese will coordinate the Los Angeles team.
{\bf Raghavan} was recently hired from ICSI and has worked with rural networks in Mendocino county and in  Google data centers.  {\bf Netravali} is a recent UCLA hire from MIT where his PhD. dissertation focussed on debugging complex web systems; he plans to apply some of his insights in distributed systems to debugging networks from an application perspective. Finally, {\bf Tamir} has extensive experience with switching hardware (he is the co-inventor of the well known Wavefront algorithm for scheduling crossbars that is used in PMC Sierra switches), computer architecture and testing. Tamir has been working with Millstein and Varghese on network operation for the last year,
and provides a unique systems perspective on testing from an OS and Architecture
perspective.
