\subsection{From Constraints to Wiring: Topology Synthesis}

\paragraph*{Tool:} (Challenge C4, Govindan, Millstein, Raghavan). A network
architect must map a specification of a network in terms of
constraints (that reflect both design objectives and resource limits)
into a physical topology of routers and wires ({\bf S4}).
%
\paragraph*{Science:} Past work~\cite{condor} has focused on {\em data
  centers} and {\em connectivity} constraints and used mapping
heuristics; we will generalize to {\em rurall networks} and CDNs with {\em
  generic} constraints.


%early versions focused on
%connectivity~\cite{PA} and structure~\cite{Hongsuda}, and later
%versions explored bandwidth, latency and cost~\cite{HOT}.

\paragraph*{Approach:}
%
Network topologies must balance competing objectives including high
bandwidth, fault tolerance and low cost. Today, designing a large
network topology can take a human expert up to six weeks. Although
modeling and simulation tools are often used as aids, the design
process is still largely done by hand.  While topology design can be
framed as combinatorial optimization~\cite{PA,
  Hongsuda,HOT}, prior work has only been used to {\em understand}
the structure of existing networks and not to {\em design} new ones.

We plan to develop a domain-specific programming language in which
planners may specify constraints on the design.  Our point of
departure is recent work at Google on a domain-specific language (DSL)
for data-center topologies called TDL~\cite{condor}). TDL can be used
to generate a specific class of topologies (Fat-Trees~\cite{FatTree})
whose links and switches are relatively homogeneous, subject to
connectivity constraints between different tiers of the Fat-Tree. We
will design a more general DSL for wide-area networks (WAN) and
enterprise networks that will support more general constraints
essential for these networks such as latency, traffic-carrying
capacity, costs, and limited optical touchdown points available from
optical vendors.

%many more dimensions in the specification language:
%higher-level constraints on end-to-end latency and link capacity
%oversubscriptions, constraints expressing commercially available
%router capabilities and link capacities (since WAN and enterprise
%topologies are constructed from heterogeneous devices and link
%capacities), constraints on the cost of wide-area links, constraints
%that simplify the manageability of the network (e.g., hierarchical
%structure that permits designers to easily visualize network
%structure) and constraints that specify optical touchdown points
%available from optical network providers.  The language will come
%equipped with tools for synthesizing the missing pieces of a design
%and for analyzing its qualities---e.g., a planner may wish to debug a
%design by checking if specific properties are satisfied.

We plan to use Counter-Example Guided Inductive Synthesis
(CEGIS)~\cite{sketch} to search through the space of possible
topologies, learning from counter-examples to quickly rule out
topologies that do not provide required functionality. To make
synthesis tractable, we will need to build tools that can rapidly
answer reachability queries as well as quantitative questions
involving bandwidth and fault tolerance. Finally, we will explore
domain-specific heuristics such as generating topologies from a fixed
class and exploiting its structure and symmetries, inspired by
heuristically-optimal topologies~\cite{HOT}.  We will also explore
synergies with constraint-based approaches used in EDA for high-level
synthesis~(e.g., \cite{cong}).

\iffalse
Finally, we will explore several novel directions in topology
synthesis. Today, the topology planning team at Facebook uses real
data about bandwidth and latency but assumes that the routing is
optimal, when in practice traffic is carried along a relatively small
number of shortest paths. Designing topology with realistic routing
models can result in networks with higher utilization and lower
overall cost.  We also plan to explore how to synthesize topologies
that satisfy all other constraints and are also robust with respect to
expected failure models. This is a step beyond the state of the art in
large providers, who only have tools to determine whether a given
topology can satisfy a given traffic matrix under a given failure
model.\jnf{The distinction between expected failure model vs. given
  failure model being used here is a bit subtle.} Finally, we plan to
explore ways of incrementally augmenting designs in ways that preserve
the original design constraints (or, in addition, new performance
properties) while requiring minimal modifications to the existing
network.


\paragraph*{Task.} (Challenge C2,  Govindan,Millstein, Raghavan, ). After designing
router hardware, a network architect must wire routers together to
make a physical network ({\bf S4}).
%

Network topologies are expected to balance a number of competing
factors including high bandwidth, low latency, fault tolerance,
incremental extensibility, and low cost. Today, designing a large
network topology can take a human expert up to four to six
weeks. Although modeling and simulation tools are often used as aids,
the design process is still largely done by hand. One factor that
makes topology design challenging is that there are intricate
dependencies between the structure of the topology, traffic demands,
and routing algorithms. There is also a significant amount of
uncertainty about these inputs---e.g., devices and links can fail
unexpectedly, and large traffic bursts can arrive at the ingress
without warning. Cost is a key consideration: fiber-optic link across
the wide area are expensive. 

\paragraph*{Approach:}
%
We plan to create new mechanisms for designing network topologies
based on synthesis. Using the tools we will develop, a designer will
be able to specify the high-level structure and properties of the
network, and a back-end solver will automatically construct a topology
that meets those requirements. We are inspired by tools that automate
significant aspects of hardware design, as well as by recent efforts
to synthesize datacenter topologies (e.g., Condor~\cite{condor}). Our
work will significantly extend this line of work by producing tools
that can be used in a variety of settings including wide-area,
optical, and enterprise networks.

At a technical level, we plan to design domain-specific languages that
can capture the high-level constraints about the structure and
properties of the network, and use Counter-Example Guided Inductive
Synthesis (CEGIS)~\cite{sketch} to search through the space of possible topologies,
learning from counter-examples to quickly rule out topologies that do
not provide required functionality. To make synthesis tractable, we
will need to build tools that can rapidly answer reachability queries
as well as quantitative questions involving bandwidth and fault
tolerance. We will also investigate tools in which topology and
routing are designed in tandem. By delaying decisions such as using
particular paths to carry traffic, it should be possible to produce
better overall designs. Finally, we will explore domain-specific
heuristics such as generating topologies from a fixed class, and
exploiting structure and symmetry when possible.

\paragraph*{Research Challenges.}
%
Mathematically, topology design can be framed as a straightforward
combinatorial optimization problem: find a graph that provides the
required connectivity, bandwidth, latency, etc. while minimizing
cost. However, there are a number of practical hurdles that need to be
overcome to actually build a tool that solves these problems.

\textit{Domain-Specific Language.}
%
Beyond mathematical optimality, simplicity and extensibility are
important consideration in topology design. After a design is
generated, it must be realized by human operators who implement the
design by connecting actual wires on physical devices. Doing this
correctly can be challenging, especially if the design is unstructured
and complex. In practice, most designs today exhibit significant
amount of structure---e.g., hierarchy and symmetry which dramatically
simplifies the design. 
%
We plan to develop a domain-specific programming language in which
planners may specify constraints on the design. Existing work on
domain-specific languages for data-center topologies (such as
TDL~\cite{condor}) can be used to generate a specific class of
topologies (Fat-Trees) whose links and switches are relatively
homogeneous, subject to low-level constraints on connectivity between
different tiers of the Fat-Tree. A more general DSL for WAN or
enterprise networks would have to support many more dimensions in the
specification language: higher-level constraints on end-to-end latency
and link capacity oversubscriptions, constraints expressing
commercially available router capabilities and link capacities (since
WAN and enterprise topologies are constructed from heterogeneous
devices and link capacities), constraints on the cost of wide-area
links, constraints that simplify the manageability of the network
(e.g., hierarchical structure that permits designers to easily
visualize network structure) and constraints that specify optical
touchdown points available from optical network providers.  The
language will come equipped with tools for synthesizing the missing
pieces of a design and for analyzing its qualities---e.g., a planner
may wish to debug a design by checking if specific properties are
satisfied.

\textit{Co-Design of Topology and Routing.}
%
One issue is that there is a mismatch between how different aspects of
the network are modeled during the design process. For example, the
topology planning team at Facebook uses real data about bandwidth and
latency, but assumes that the scheme is optimal, when in practice it
traffic is carried along a relatively small number of shortest
paths. We plan to develop a framework that allows the topology and
routing scheme to be co-designed together. By delaying decisions about
how the network is structured and which paths are used to carry
traffic, we should be able to produce better designs than if these
problems are solved independently. Achieving this will require
enriching our domain-specific language with additional interfaces to
allow information to be passed between routing and topology design. 

We also plan to explore the important issue of fault tolerance: how to
synthesize topologies that satisfy all other constraints and are, in
addition, robust to expected failure models. This is a step beyond the
state of the art in large providers, who explore whether a given
topology can satisfy a given traffic matrix under a given failure
model.

\textit{Extensibility.}
%
Even the most carefully designed topologies must eventually be
updated. Hence, it is also important that a topology design framework
support incremental evolution---e.g., in a datacenter, operators need
to be able to bring up new racks while in the wide area, they may need
to expand into an entire new region without disrupting existing
traffic. We plan to explore ways of incrementally augmenting designs
in ways that preserve the original design constraints (or, in
addition, new performance properties) while requiring minimal
modifications to the existing network. Incrementable evolvability fits
in well with our inductive synthesis approach---essentially it amounts
to treating generated topologies as intermediate nodes in an ongoing
design process.

\paragraph*{Scientific Impact.}
%
If successful, this work will be transformative: rather than having to
design topologies by hand, using imprecise models and partial tool
support, network planners will be able to automate many of the tedious
design tasks using automated tools. This should allow them to work
more efficiently and to focus on deeper problems, which are often
ignored today, such as robustness. Our work also will have impact on
the inter-disciplinary scientific communities involved in building
these tools. Networking researchers will help identify the right
abstractions for a domain-specific language for topology
design. Programming language researchers will help implement and
optimize that language. And formal methods researchers will develop
new back-end solvers that work with quantitative domains and
incorporate network-specific optimizations and incremental modes of
operation.
\fi
