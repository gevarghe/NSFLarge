\documentclass[11pt]{article}

\topmargin=-0.5in
\oddsidemargin=0in
\evensidemargin=0in
\textwidth=6.5in
\textheight=9.0in

\begin{document}

\begin{center}
 \begin{large}
Facilities, Equipment, and Other Resources \\
\end{large}
\end{center}


Profs.\ Varghese, Millstein, Netravali and Tamir are members of UCLA's
Computer Science Department.  Their offices are just down the hall from one another.  Varghese and Millstein share a lab for their students, and Tamir and Netravali's students are in a nearby lab.  Each lab has
approximately 700 square feet of space and
includes desk space and state-of-the-art workstations for each
student.
All workstations are
connected to the Computer Science Department's local area network (LAN),
which in turn connects to UCLA's main network and from there to the broader Internet. 

The PIs also have several high-end computers in the
department's machine room, which function as compute servers and as
repositories for their research infrastructure.


In addition to its major
networking function, the department computing facility also provides many
computing resources, including disk storage and backup, general-purpose timesharing access to multi-user systems, email, and more than 600 workstations for research
and educational usage.

We hope to take advantage of a FPGA testbed built by our collaborator Ashwin Gumaste at IIT Mumbai. The Gumaste test-bed is sufficiently large, with a combined switching capacity of 3 Tbps – typical of a tier-2 network that one would find in a rural/regional setting. The test-bed is subject to services that are provisioned through a home-grown controller, which is a modification of OpenDaylight, with improvements that are essential towards including the concepts of megalabels and source-routing. 

Gumaste's team has developed 3 kinds of SDN whiteboxes as part of their test-bed: a metropolitan edge SDN box that has 10x1Gbps and 2x10Gbps IOs, a regional cross-connect whose individual stackable element has 4x10Gbps and 8x1Gbps IOs, and a core whitebox that support 72x10Gbps IOs, with capability to also support 4x2x100Gbps IOs. The edge box is built around a Virtex 6 FPGA (240T-1) and an AMCC PHY which multiplexes 3.125Gbps IO lines from the FPGA into 10Gbps IOs. The regional SDN platform is based on a Virtex 6 365T FPGA along with 4 Pemaquid OTN capable ASICs. The larger core platform has 2 switching cards, each supporting multiple Virtex 7 690T FPGAs, and up to 12 IO cards, each supporting 12x10Gbps or 2x100Gbps (2 slot width) with each IO card built around two parallel units of Virtex 7 FPGAs. All the FPGA based hardware interacts with a Java-based network management system that consists of a provisioning module, a telemetry module and a network discovery module, together which imitates an SDN controller. 

We plan to use the Gumaste testbed as part of an end to end demonstration that shows routing and scheduling synthesis together with an application such as remote telesurgery.

\end{document}
