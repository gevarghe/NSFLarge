\centerline{\large\textbf{Collaborative Research: Network Design Automation}}
\bigskip

When we write a document using Office365, order flowers from Amazon, or update your Facebook status, you are accessing a world of rich services via a network. Network reliability is paramount for business, science, and government because downtime affects productivity, economic output, and social communication. However, reliability is difficult to achieve in the face of rapid technology evolution, increased network programmability, manual configuration, and the stringent economics of the IT industry.

This grant seeks to take first steps towards creating a new field, \emph{Network Design Automation} (NDA), which seeks to create computer-aided design (CAD) tools that reduce design effort and increase reliability based on a scientific understanding of network structure and behavior. We are inspired by the Electronic Design Automation (EDA), a \$7B industry that underpins the \$100B chip industry and is also a vibrant intellectual discipline in its own right.  While we will build on 
the last five years of progress in network verification done by the community (that has already resulted in startups like
Veriflow Networks and Forward Networks), we seek to extend network verification with tools like specification mining
and control plane testing.  We also aim to forge new directions in network scripting to lower the barriers for operators, topology design for rural networks, tying application performance to network problems using machine learning, and 
understanding failure logs using Natural Language Processing.  NDA will make it possible to construct and manage networks that are guaranteed to satisfy strong formal properties, while also minimizing the frequency of failures and greatly reducing recovery time.

UCLA will serve as the host institution. UCLA (Varghese: Networking; Millstein: Programming Languages, Netravali, Tamir: Systems) will be responsible for tools and theory for specification mining, debugging, and testing, while USC  (Govindan: Networking, Ragahavan: ICDT) will be responsible for fault diagnosis, rural network design, and scripting for operators.  However, as we are all based in Los Angeles we will all work on all problems together with regular meetings and co-advising students between the two institutions to expose syngergies between each research topic.  While 6 is a large number of PIs, Varghese at UCLA and Govindan USC will be responsibe for coordinating individual research teams into a coherent whole.

\smallskip
\noindent\textbf{Intellectual Merit.}~
%
Networks are more complex in several dimensions than either programs or chips, including scale ($10^5$ devices and zetabytes of traffic), heterogeneity (devices, protocols, vendors), management complexity (most configuration is still at "assembly" level), rate of evolution (new data centers and service rollouts), and high availability needs (<40 minutes per month downtime for $3 \times 9$s availability). Thus, we need better abstractions and tools across the board: more expressive languages that can express richer functionality, and more powerful tools that can efficiently analyze and synthesize configurations at scale. To provide the scientific underpinning for these efforts, we will develop new mathematical frameworks for reasoning about network structure and behavior. As with recent progress in network verification, we seek to harness
interdisciplinary approaches.  Besides a consistent focus on using algorithm search (e.g., SAT solvers for say rural topologies),
we also explore the impact of data mining (to learn specifications), machine learning (to build associations for debugging),
and NLP (to mine operator logs).  As with our earlier work in network verification, we do not seek to merely reuse existing
mechanisms from these fields (e.g., SAT solvers, K-means culstering) but to exploit domain-specific insights about the {\em structure} of modern networks to create more scalable tools{\em  and} new intellectual areas.

\smallskip
\noindent\textbf{Broader Impacts.}~
Our scientific discoveries and artifacts developed will have broad impact by transforming both academi, industrial and rural approaches to networking:
(1) Public Clouds: We will partner with Microsoft (via Albert Greenberg) and Google (via Amin Vahdat), both of whom are former colleagues of Varghese and Govindan to develop, deploy, and evaluate NDA technology for clouds. (2) Barath Raghavan will work with his colleagues at rural networks like Motech) and African Networks like Grameen via his colleague David Hutchful to help deploy topology design and configuration scripting tools (3) We will release a set of configuration scripting tools for
specialized networks like home networks as well as rural networks. (4) Education and Students: Together with our colleagues at Cornell and Princeton, we will continue to build a sustaining community of students who will drive the field forward via new courses that can be used and extended by the community.
