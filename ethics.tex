\section{Intellectual Property and Ethics}

The main artifacts produced by this expedition will be research
papers, software tools, and an electronic textbook with accompanying
teaching materials. We will create a project website with links to
these artifacts. To encourage broad dissemination of our results, we
will seek to publish in conferences, journals, and public
respositories with reasonable copyright and open access policies
(e.g., arXiv, ACM, USENIX, etc.) and will release our electronic
textbook under an open-source license (e.g, Creative Commons). We will
encourage project personnel to make software artifacts available under
open-source licenses (Apache, BSD, GPL, etc.) to allow others to build
on, adapt, and extend our work.

Although the research proposed here is primarily concerned with
developing algorithms and tools, because networks play a critical role
in modern society, there is also an implicit human dimension to our
work. We will provide training for PhD students and postdoctoral
resarchers on proper research conduct, paying special attention to
specific issues related to network infrastructure---e.g., during an
emergency, networks provide a critical line of communication for
public safety officers, so ensuring that the network is reliable may
literally save lives. We will seek out instutitional resources for
guidance on any aspects of our research that impinge on further
ethical issues.
